\chapter{Introduction}

Most of the malls and shops have scarcity of resources. They have limited number of shelves and space, thus can show the 
customers only a limited fraction of choices available. On the contrary to it online stores and e--shopping provides more options to the user to choose from. This abundance of choices and need of custom--tailoring the options to the needs of specific users make recommendation fairly important for e--commerce. It won't be wrong to say that it is the backbone of online e-commerce. We have a number of recommenders which use various techniques like content--based or collaborative filtering which recommend similar items based on user's previous choices and requirements. So we buy a Denim Jeans, we will be recommended with various Jeans ranging in various prices, companies, designs or other aspects. But we have rarely seen an online garment seller who recommends matching tops, t--shirts or accessories when we buy Denim jeans. Many times while shopping for garments online users find difficulty in finding clothes that match whatever is already in the cart. In such situations a recommendation can influence and increase the sales, and also enhances customer loyalty and engagement. The attractiveness of garments in this situation depend on other items bought by the user. Therefore it is beneficial to consider a whole set of recommendation instead of treat items individually. In this project we target this problem of online garment sellers, where given an item in the shopping cart we intend to suggest items complementary to it which may contain garments or accessories which make a complete set as per current fashion. We call it \emph{Complete the Look problem}. This falls under the class of recommendation problem called the \textit{Bundle Recommendation} problem.

\section{Bundle Recommendation}
Bundle is referred to as a set of items that the customer considers or buys together. Since we are specifically working with garments, a bundle in our case will be a set of clothes that are complementary to each other and  can be worn together. In classical marketing research it is well known that there is a `$1+1>2$' effect in carefully designed product bundles\cite{bundleReco}. This is because they generate context for the customer to buy more products. For example, if someone is buying a suit, it is irresistible to buy a tie that goes well with it. We will recommend a bundle of matching clothes which will lead to an increased sales probability.

\section{Dataset and Ground-truth}
Most progress in computer vision has been partly possible because of annotated public datasets. Obtaining large datasets with reasonably clean dataset is a daunting challenge. Specially annotated images of our requirements which express social, cultural and commercial importance are rarely available. We would like to collect fashion images which contain a full view of street fashion where model has several fashion accessories. The first large scale public scale dataset for street fashion was presented in \cite{clothParsing}. It had 158235 images out of which only 685 were annotated. Another similar dataset was introduced in \cite{largeScaleReco}. They called it the Fashion--136K. This consists of fashion images along with other metadata such as annotations of accessories, brands and demographic of the fashionistas who posted the images. We can use the fashion websites as source of annotated data for fashion recommendation. There are various websites where users and fashionstas upload their images in various clothes which in accordance to the current fashion trends, and this is a potential source of large fashion recommendation datasets. The data is easily available and is user annotated.

Our problem in hand i.e. recommending a bundle of complementary items has has a couple of computational challenges:

\begin{enumerate}
\item Representation of a fashion items as visual features is an open ended problem. Not much work has been done on it. Mostly color templates or HSV histograms are used.
\item Developing the notion of relative similarity is another challenge. It is subjective and we don't have concrete numerical measure for it. It depends on various factors like location, season and sometimes the occasion on which the clothing is to be used. This makes learning inherently complex.
\end{enumerate}

Apart from them there are a number of practical challenges that are to be addressed like quality of input and training data, scalability issues in terms of memory and speed requirements. The main purpose of the project can be viewed in Figure 1.1.

We formulate the recommendation problem as follows: given an image $i$ containing `$k$' fashion items, referred to as `part features'(eg. red top, black boots, etc.), we describe the image $P_i$ as $P_i^{T} := [p_{i1}, p_{i2}, ..., p_{ik}]$ where each $p_{ij}$ are textual part features which are two-tuples of part category and description. We learn a model from our dataset of fashion images, say \textbf{P}, where \textbf{P} := $[P_1, P_2, ... P_n]^{T}$. The task of our recommendation system is, given one or more apparel, and corresponding part features $p$'s as input query, recommend garments which can be worn with it/them as a set.

%even more text\footnote{<footnote here>}, and even more.
